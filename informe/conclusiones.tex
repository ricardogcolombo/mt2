Luego de la experimentación y análisis de los resultados, concluimos el método de calculo basado en \textbf{CMM} es mas justo en el 
caso de torneos donde los equipos no juegan la misma cantidad de partidos y donde el empate no es una opción. Ya que asigna un puntaje 
en base no solo a los resultados obtenidos, sino contra quien fueron obtenidos. Obteniendo un ranking basado en la meritocracia del resultado. \\

Para el caso de torneos donde cada equipo juegue la misma cantidad de partidos el método de \textbf{WP} a nuestro criterio resulta mas justo. 
Debido a que todos se pusieron a prueba la misma cantidad de veces. \\

Respecto a que implementación de \textbf{CMM} resulta mas eficiente. La conclusion es que depende. Ambas obtienen el mismo resultado, 
la principal ventaja de \textbf{Cholesky} es que realiza menos computos, mientras que la de \textbf{Eliminación Gaussiana} es que es mas 
sencilla su implementación. \\

Por último sobre \textbf{La utilización de técnicas avanzadas de análisis de datos son imprescindibles para mejorar cualquier deporte}, 
consideramos que la frase no es del todo cierta. Afortunadamente la frialdad de los números no es aplicable a la pasión de todos los deportes. 
Mientras en contados deportes el resultado puede predecirse de antemano, debido a las caracteristicas de los rivales, como en el caso del Polo, 
esta analogía no puede aplicarse a deportes como el Fútbol donde en innumerables ocasiones el equipo menos favorito termina llevandose el partido.
