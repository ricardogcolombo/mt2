Luego de la experimentación y análisis de los resultados, concluimos el método de calculo basado en \textbf{CMM} es m\'as justo en el 
caso de torneos donde los equipos no juegan la misma cantidad de partidos y donde el empate no es una opción. Ya que asigna un puntaje 
en base no solo a los resultados obtenidos, sino contra quien fueron obtenidos. Obteniendo un ranking basado en la meritocracia del resultado. \\

Para el caso de torneos donde cada equipo juegue la misma cantidad de partidos el método de \textbf{WP} a nuestro criterio resulta m\'as justo. 
Debido a que todos se pusieron a prueba la misma cantidad de veces. De lo contrario se vuelve mas justo el CMM porque no seria justo que alguien que juega menos partidos
este mas alto en el ranking que alguien que no y eso en WP se nota cuando comienza a haber diferencias de partidos.t

Respecto a que implementación de \textbf{CMM} resulta igual de eficiente con ambas obtienen el mismo resultado con lo cual no notamos diferencias para este caso en cuanto
preferencias.

Por último sobre la frase \textbf{La utilización de técnicas avanzadas de análisis de datos son imprescindibles para mejorar cualquier deporte}, 
 y en base a las peliculas vistas , consideramos que el Analisis de datos es una herramienta importante para mejorar el rendimiento deportivo de un equipo, 
siempre y cuando se utilizen correctamente. 
Las estadisticas son indicadores de que tan bien rinden los jugadores pero hay factores externos como el animo, el nivel de los competidores 
que no son considerados en las estadisticas ya que no son medibles y pueden modificar bastante el rendimiento de los jugadores.
Afortunadamente la frialdad de los números no es aplicable a la pasión de todos los deportes. 

Mientras en contados deportes el resultado puede predecirse de antemano, debido a las caracteristicas de los rivales, como en el caso del Polo, 
esta analogía no puede aplicarse a deportes como el Fútbol donde en innumerables ocasiones el equipo menos favorito termina llevandose el partido, aunque hay casos en donde
Sports Analitycs parece haber dado buenos resultados, sin ir m\'as lejos en el \'ultimo mundial de futbol donde gano Alemania, se utilizo Sports Analistics
con una herramienta llamada SAP Sports One con la cual se pueden saber todas las estadisticas de los jugadores y se podrian tener datos de los rivales para realizar
cierto an\'alisis previo a cada partido, para este dejamos una nota de un diario canadiense en la bibliografia (item 3 bibliografia).
