Para resolver el enunciado planteado, realizamos la implementación de dos técnicas de calculo de ranking distintas. \\

\subsection{Porcentaje de Victorias}

La primer técnica es \textbf{Porcentaje de Victorias} que a lo largo del análisis denominaremos \textbf{WP} que consiste en tomar el promedio de partidos ganados / partidos jugados. Esta técnica basicamente analiza la performance de un equipo participante en los partidos jugados. \\

En este caso el score de un equipo no es afectado por la cantidad de partidos y resultados obtenidos de los demás participantes, pero esto si afecta su posición final en el ranking. \\

Esta técnica a priori no aporta mucha informacion respectoa la posibilidad de victoria en el siguiente encuentro y tampoco considera el ranking del rival enfrentado. Ya que todos los partidos valen lo mismo. \\

La implementación consiste en calcular: \sum_{i=1}^n{} \frac{G_i}{T} donde \textbf{n} es la cantidad de partidos jugados, \textbf{G_i} corresponde a partidos ganados y \textbf{T} al total de partidos jugados. \\


\subsection{Método Matriz de Colley}

Para la implementación de esta técnica nos basamos en el paper \textbf{The Colley Matrix Explained}. La cual consiste en plantear un sistema de ecuaciones


\subsubsection{Eliminación Gaussiana}
\subsubsection{Cholesky}


