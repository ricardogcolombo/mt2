\subsection{Entrada y salida de los algoritmos}

Dados los requirimientos de la catedra el programa toma como parametros 3 argumentos, el primero es el archivo de entrada, luego el archivo de salida y
por ultimo el modo. Los modos solicitados por la catedra son:
\begin{enumerate}
    \item Eliminacion Gaussiana(EG)
\item Factorizacion de Cholesky(CL)
    \item WP
    \item Cholesky con modificacion de partidos jugados
    \item Cholesky haciendo ganar al ultimo
    \end{enumerate}
    Ademas de los 3 modos solicitados por la materia agregamos 2 mas.
    Este modo corre cholesky y luego busca 2 equipos que hayan jugado previamente para cambiar su resultado y luego volver a ejecutar cholesky
    Este modo corre cholesky y luego ejecuta un ciclo donde el objetivo es lograr que el que haya salido ultimo llegue al primer puesto ganandole
    al que tiene por arriba inmediato en el ranking. En cada paso agrega un partido mas y vuelve a calcular cholesky para la nueva matriz.
    En el momento que llega al primer puesto retorna por stdout la cantidad de partidos que ejecuto hasta llegar al primer puesto.

    Tanto el formato de entrada y de salida del programa son los solicitados por la catedra, para el archivo de entrada la primer linea tiene 2 valores n,
    que representa la cantidad de equipos  y k que representa la cantidad de partidos,
    luego se tienen k lineas donde esta el resultado de cada partido representado por 5 parametros f,e1,r1,e2,r2.
    f contiene es una fecha de caracter opcional, en nuestos experimentos esta fecha no fue utilizada, luego se tienen e1 , numero de equipo 1,
    r1 cantidad de anotaciones del equipo 1 y sus equivalentes con e2, r2 respectivamente para el equipo 2.


\subsection{Sistema a resolver}

    \[ C_{i,j} =
    \begin{cases}
        n_{i,j}       & \quad \text{si }  \text{i\neqj}\\
        2+n_i & \quad \text{si } \text{ i \eq j }\\
    \end{cases}
    \]

\subsection{Eliminacion Gaussiana(EG)}

\subsection{Factorizacion de Cholesky(CL)}

\subsection{Implementacion}
