\subsection{Entrada y salida de los algoritmos}

Dados los requirimientos de la catedra el programa toma como parametros 3 argumentos, el primero es el archivo de entrada, luego el archivo de salida y
por ultimo el modo. Los modos solicitados por la catedra son:
\begin{enumerate}
    \item Eliminacion Gaussiana(EG)
\item Factorizacion de Cholesky(CL)
    \item WP
    \item Cholesky con modificacion de partidos jugados
    \item Cholesky haciendo ganar al ultimo
    \end{enumerate}
    Ademas de los 3 modos solicitados por la materia agregamos 2 mas.
    Este modo corre cholesky y luego busca 2 equipos que hayan jugado previamente para cambiar su resultado y luego volver a ejecutar cholesky
    Este modo corre cholesky y luego ejecuta un ciclo donde el objetivo es lograr que el que haya salido ultimo llegue al primer puesto ganandole
    al que tiene por arriba inmediato en el ranking. En cada paso agrega un partido mas y vuelve a calcular cholesky para la nueva matriz.
    En el momento que llega al primer puesto retorna por stdout la cantidad de partidos que ejecuto hasta llegar al primer puesto.

    Tanto el formato de entrada y de salida del programa son los solicitados por la catedra, para el archivo de entrada la primer linea tiene 2 valores n,
    que representa la cantidad de equipos  y k que representa la cantidad de partidos,
    luego se tienen k lineas donde esta el resultado de cada partido representado por 5 parametros f,e1,r1,e2,r2.
    f contiene es una fecha de caracter opcional, en nuestos experimentos esta fecha no fue utilizada, luego se tienen e1 , numero de equipo 1,
    r1 cantidad de anotaciones del equipo 1 y sus equivalentes con e2, r2 respectivamente para el equipo 2.


\subsection{Sistema a resolver}

    \[ C_{i,j} =
    \begin{cases}
        n_{i,j}       & \quad \text{si }  \text{i\neqj}\\
        2+n_i & \quad \text{si } \text{ i \eq j }\\
    \end{cases}
    \]
Esta matriz es simetrica y definida positiva y ademas es estrictamente diagonal domimante.
Por lo tanto es posible hallar la factorizacion de Cholesky , que no es mas que un caso
de factorizacion LU , esto nos permite asegurar que a la hora de hacer eliminacion 
gaussiana no vamos a hallar 0 en la diagonal , con lo cual no hace faltar considerar
pivoteo parcial en la eliminacion Gaussiana.  

%Las Caracteristicas de esta matriz son -La Matriz es Simetrica y definida positiva
%-Es estrictamente Diagonal Domimante  
%-Por lo tanto se le puede hacer la factorizacion de Cholesky que no es mas que
%un caso particular de Factorizacion LU
%-Si se le puede hallar la factorizacion LU , eso significa que la eliminaion gausiana
%no va a hallar 0 en la diagonal , con lo cual no hace falta considerar pivoteo en el algoritmo
%-El metodo para crear la matriz esta basado en la regla de metodos sucesivos de LaPlace
    

\subsection{Método Matriz de Colley}

Para la implementación de esta técnica nos basamos en el paper \textbf{The Colley Matrix Explained}. La cual consiste en plantear un sistema de ecuaciones

La principal fortaleza de este método es que es útil para obtener rankings en torneos donde los participantes no juegan la misma cantidad de partidos. Además de que al armar el sistema en base a los resultados pasados, da relevancia al calendario de juegos de cada participante. Como se intentará demostrar en la sección de experimentos, utilizando esta tecnica importa contra quien se gana y contra quien se pierde. Además tiene el atractivo de que da una idea sobre la posibilidad de victoria en el siguiente partido, considerando los partidos anteriores. \\

La principal desventaja y que hace que no sea aplicable a muchos de los deportes es que los empates no pueden ser modelados. \\

Una vez obtenida la \textbf{Matriz de Colley} vamos a  presentar dos técnicas de resolucion del sistema de ecuaciones obtenido. \\

\subsubsection{Eliminación Gaussiana}

La implementación del algorimo de \textbf{Eliminación Gaussiana} que elegimos es la que se encuentra en el libro \textbf{Burden}. \\


\begin{algorithm}
\begin{algorithmic}[1]\parskip=1mm
\caption{void Gauss(matriz A, vector b)}
  \STATE{Para k=$1...n$}
  \STATE{\quad Tomo el elemento $a_{k,k}$ como pivot}
  \STATE{\quad Para $i = k+1,...k+p$}  
  \STATE{\quad \quad Para $j = k+1,...i+p$}
  \STATE{\quad \quad \quad $a_{i,j}  = a_{i,j} - a_{k,j} * (a_{i,k} / a_{k,k})$}
  \STATE{\quad \quad $a_{i,k} = 0$}
\end{algorithmic}
\end{algorithm}


\subsubsection{Cholesky}

La implementación del algorimo de \textbf{Eliminación Gaussiana} que elegimos es la que se encuentra en el libro \textbf{Burden}. \\


\begin{algorithm}
\begin{algorithmic}[1]\parskip=1mm
\caption{void Cholesky(matriz A, vector b)}
  \STATE{l_{1,1} = \sqrt{a_{1,1}}}
  \STATE{Para $j = 2,...n$}
  \STATE{\quad l_{j,1} = a_{j,1} / l_{1,1}}  
  \STATE{\quad Para $i = 2,...n-1$}
  \STATE{\quad \quad $l_{i,i}  = (a_{i,i} - \sum_{k=1}^{i-1}{l^2_{i,k}}^{1/2})$}
  \STATE{\quad \quad Para $j = i+1,...n1$}
  \STATE{\quad \quad \quad $l_{j,i}  = (a_{j,i} - \sum_{k=1}^{i-1}{l_{j,k} l_{i,k}} / l_{i,i})$}
  \STATE{\quad $l_{n,n}  = (a_{n,n} - \sum_{k=1}^{n-1}{l^2_{n,k}}$}
\end{algorithmic}
\end{algorithm}


\subsection{Porcentaje de Victorias}

La primer técnica es \textbf{Porcentaje de Victorias} que a lo largo del análisis denominaremos \textbf{WP} que consiste en tomar el promedio de partidos ganados / partidos jugados. Esta técnica basicamente analiza la performance de un equipo participante en los partidos jugados. \\

En este caso el score de un equipo no es afectado por la cantidad de partidos y resultados obtenidos de los demás participantes, pero esto si afecta su posición final en el ranking. \\

Esta técnica a priori no aporta mucha informacion respectoa la posibilidad de victoria en el siguiente encuentro y tampoco considera el ranking del rival enfrentado. Ya que todos los partidos valen lo mismo. \\

La implementación consiste en calcular: \sum_{i=1}^n{} \frac{G_i}{T} \\

Sonde \textbf{n} es la cantidad de partidos jugados, \textbf{G} corresponde a partidos ganados y \textbf{T} al total de partidos jugados. \\
