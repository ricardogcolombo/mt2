\documentclass[a4,11pt]{article}

\parindent=10pt
\parskip=6pt
%\usepackage[width=15.5cm, left=2.5cm, top=2cm, height= 24.5cm]{geometry}

\usepackage[paper=a4paper, left=2cm, right=2cm, bottom=2.5cm,top=2.5cm]{geometry}

% Paquetes de nacionalización. No olvidar para poder poner tildes!
\usepackage[spanish]{babel}
\usepackage[utf8]{inputenc}

% Paquetes para graficos
\usepackage{subfig}
% \usepackage{graphicx} %% La caratula lo incluye

% Paquetes para matematica
\usepackage{amsmath}
\usepackage{amsfonts}
\usepackage{amssymb}

% Paquetes para pseudo
\usepackage{algorithm}
\usepackage{algorithmic}

% Caratula (Recordar logo_uba.jpg y logo_dc.jpg)
\usepackage{caratula}

% Paquetes para tablas
\usepackage[table]{xcolor}

% Se pueden sacar?
\usepackage{url}
\usepackage{float}
\usepackage{afterpage}
\usepackage{tabularx}

% Color de links
\usepackage{hyperref}
\hypersetup{
    colorlinks,
    citecolor=black,
    filecolor=black,
    linkcolor=black,
    urlcolor=black
}

\begin{document}


\materia{Metodos numericos}
\submateria{Primer Cuatrimestre de 2016}
\titulo{Trabajo Pr\'actico 1}
\subtitulo{“(No) Todo Pasa”}
\integrante{ Leonardo Raed}{579/04}{leo\_raed@yahoo.com}
\integrante{ Ricardo Colombo}{156/08}{ricardogcolombo@gmail.com}
\integrante{ Diego Santos}{874/03}{diego.h.santos@gmail.com}
\integrante{ Luis Badell }{246/13}{luisbadell@gmail.com}


\maketitle
\pagebreak
  
\tableofcontents

\pagebreak

\section{Introducción teórica}


El objetivo de este trabajo es la realización y el análisis de algoritmos eficientes para el reconocimiento óptico de caracteres (OCR), particularmente de dígitos,  a través de la utilización de técnicas simples de Machine learning.
\\
El trabajo consiste en una serie de experimentaciones. El desarrollo de estas encuentra un hilo conductor en las mejoras aplicadas a un algoritmo basadas en problemas particulares que se pueden encontrar en la resolución del problema:

\begin{itemize}

    \item Se parte de una base de datos de imágenes ya etiquetadas y otra con imágenes sin etiquetar. Usando la base de datos etiquetada como información de entrenamiento del algoritmo, se intenta etiquetar de modo correcto los dígitos de la base de datos sin etiquetas.

    \item La primera aproximación a la resolución del problema utiliza el método más intuitivo encontrado: Por cada imagen de la base de datos sin etiquetas, se busca la que más se le parece en la base de datos etiquetada y se marca a la imagen sin etiqueta con la etiqueta de aquella que denominamos como la más parecida. Por supuesto, todavía queda determinar cual es el criterio para decir que dos imágenes se "parecen". Esta definición está dada con profundidad en la sección de desarrollo.

    \item Surge entonces la pregunta acerca de que pasa si, por una particularidad de la imagen, la etiqueta más parecida no es la correcta para el dígito a averiguar. Para mitigar este problema parcialmente se pueden tomar las $k$ imágenes más parecidas (que a partir de ahora llamaremos vecinos) y elegir como etiqueta aquella que se repita más entre los $k$ vecinos. Detrás de esta idea se encuentra el algoritmo $KNN$, que se utiliza para mejorar el comportamiento en estos casos donde el vecino más cercano no pertenece necesariamente a la misma clase que la imagen a etiquetar.

    \item A esta idea se le puede aplicar una mejora sustancial utilizando un método probabilístico conocido como $PCA$. Este consiste en aplicar una transformación a las imágenes, de tal manera de solo tener en cuenta aquellas de mayor variabilidad y desechar aquella información que pueda estar introduciendo ruido.

    \item Por ultimo,  con una idea a $PCA$,  utilizaremos el metodo $PLS-DA$ con la diferencia de utilizar informacion original para realizar la transformacion.


\end{itemize}
Para entender las diferencias y similitudes entre los métodos y sus variantes, se realizan los experimentos con variaciones en los parámetros. En el caso de
$KNN$ se varía la cantidad de vecinos, esto ayuda a entender que valores ayudan a la optimización del algoritmo.
\\
Para el caso de la mejora utilizando el algoritmo de $PCA$ también hay que tener en cuenta el $\alpha$ utilizado. Vamos a ver como modificar este valor
conlleva diferentes tiempos de ejecución y pérdida o ganancia de precisión, y en cuanto al metodo de $PLS-DA$  vamos a variar el valor de gamma.




\pagebreak
\section{Desarrollo}
Para resolver el enunciado planteado, realizamos la implementación de dos técnicas de calculo de ranking distintas. \\

\subsection{Porcentaje de Victorias}

La primer técnica es \textbf{Porcentaje de Victorias} que a lo largo del análisis denominaremos \textbf{WP} que consiste en tomar el promedio de partidos ganados / partidos jugados. Esta técnica basicamente analiza la performance de un equipo participante en los partidos jugados. \\

En este caso el score de un equipo no es afectado por la cantidad de partidos y resultados obtenidos de los demás participantes, pero esto si afecta su posición final en el ranking. \\

Esta técnica a priori no aporta mucha informacion respectoa la posibilidad de victoria en el siguiente encuentro y tampoco considera el ranking del rival enfrentado. Ya que todos los partidos valen lo mismo. \\

La implementación consiste en calcular: \sum_{i=1}^n{} \frac{G_i}{T} donde \textbf{n} es la cantidad de partidos jugados, \textbf{G_i} corresponde a partidos ganados y \textbf{T} al total de partidos jugados. \\


\subsection{Método Matriz de Colley}

Para la implementación de esta técnica nos basamos en el paper \textbf{The Colley Matrix Explained}. La cual consiste en plantear un sistema de ecuaciones


\subsubsection{Eliminación Gaussiana}
\subsubsection{Cholesky}





\pagebreak
\section{Experimentaci\'on}
Para analizar la efectividad y ecuanimidad de esta nueva forma de calcular el ranking vamos a realizar una serie de test a fin de obtener un analisis cuantitativo y cualitativo que nos permita compararlo con el clásico método de \textbf{WP}. \\
Con los test esperamos encontrar ventajas y desventajas de esta forma de medición, particularmente en escenarios donde no todos los participantes juegen la misma cantidad de partidos.
\\

Además realizaremos una comparación de los métodos de \textbf{Eliminación Gaussiana} y \textbf{Cholesky} para ver cual de los dos computa los rankings de manera mas eficiente.
\\

En esta sección solo presentaremos los experimentos realizados y los resultados obtenidos. Las conclusiones de cada experimento
las presentaremos en la seguiente sección. 


\subsection{Ranking}

Vamos a comparar la tabla de ranking obtenida a partir de un set de datos de la \textbf{ATP 2007}. Es decir calculamos el ranking a partir de la técnica \textbf{WP}, considerando partidos ganado / partidos jugados, a pesar de que no todos los jugadores hayan participado de la misma cantidad de partidos. Comparandolo con el \textbf{CMM} implementado con \textbf{Eliminación Gaussiana} y \textbf{Cholesky}. \\


\begin{figure}[H]
\centering
\includegraphics[width=1\textwidth]{IMG/Comparativa WP- CMM todos.png}
\caption{Rankings Calculados con las 3 tecnicas}
\label{fig:Comparacion de tecnicas}
\end{figure}

\\

Hemos realizado un zoom dentro del grafico para corroborar el nombre y posicion de ambos rankings.
\\

\begin{figure}[H]
\centering
\includegraphics[width=1\textwidth]{IMG/comparativa WP - CMM zoom.png}
\caption{Zoom de las primeras posiciones}
\label{fig:Zoom de las primeras posiciones}
\end{figure}

\\

\subsection{¿Importa a quien se le gana?}


En el escenario que se utilize \textbf{WP} realmente no importa a que equipo se le gane, ya que todos los partidos tienen la misma importancia y se les asigna el mismo puntaje. Pero en el caso de \textbf{CMM} resulta mas interesante plantearse esta pregunta. \\

La hipótesis que tenemos es que tomando un equipo de mitad de tabla, que denominamos \textbf{medio} el hecho de que le gane al lider de la tabla va a mejorar mucho mas el ranking que derrotando al que ocupe la última posición. \\

Realizamos un test tomando al equipo \textbf{medio}, y agregando un partido victorioso contra el puntero y analizamos como se modifica su ranking. Luego tomamos la tabla inicial, es decir sin ganarle al puntero, y repetimos el experimento esta vez derrotando al último. \\

Presentamos los resultados obtenidos.

\\


******Aca van los graficos de: tabla inicial, perder contra el primero y perder contra el ultimo (sacando el partido con el primero)

\\




\subsection{¿Importa contra quien se pierde?}

Para verificar si importa contra que equipo se juega, proponemos el test de tomar un equipo que se encuentra por la mitad de la tabla. Por notación denominamos a este equipo como \textbf{medio}. \\

Para lograrlo calculamos un ranking a partir de un set de datos. Y luego generamos una nueva instancia enfrentando a \textbf{medio}contra el actual puntero y calculamos el nuevo ranking. Volvemos a tomar la primer instancia y lo enfrentamos contra el último, calculamos nuevamente el ranking y luego comparamos los tres rankings obtenidos. La premisa que tenemos del experimento es que el ranking del equipo \textbf{medio} no debería ser afectado por el rival contra el que perdió.\\

Este experimento lo calculamos usando la técnica de \textbf{CMM}, ya que considerar \textbf{WP} no afecta el resultado.

\\
A continuación presentamos los graficos obtenidos.

\\


******Aca van los graficos de: tabla inicial, perder contra el primero y perder contra el ultimo (sacando el partido con el primero)

\\



\subsection{Racha ganadora}

Para estudiar la ecuanimidad del \textbf{CMM} realizamos un experimento tomando al participante del \textbf{ATP 2007} que se encontraba en el último puesto y le asignamos una racha ganadora contra los primeros diez jugadores del ranking. \\

Además este test nos permite observar como la racha de un jugador afecta al ranking global y si ganandole a los mejores realmente escala una considerable cantidad de posiciones en el ranking. \\

A continuación presentamos el ranking calculado construido de la siguiente manera: \\

\begin{itemize}
	\item Eje X el valor obtenido al ejecutar CMM implementado con Cholesky.
	\item Eje Y el valor obtenido al ejecutar CMM implementado con Eliminación Gaussiana.
	\item El tamaño de la burbuja es la cantidad de partidos jugados.
\end{itemize}

\\

\begin{figure}[H]
\centering
\includegraphics[width=1\textwidth]{IMG/comparativa cmm -cmm foto 0.png}
\caption{Zoom de las primeras posiciones}
\label{fig:Zoom de las primeras posiciones}
\end{figure}

\\
Y a continuación el ranking después de los 10 partidos: \\
\\
En el gráfico se observa mediante una flecha cual es la ganancia en Ranking que tiene el ultimo jugador al ganarle a los top 10.\\

\begin{figure}[H]
\centering
\includegraphics[width=1\textwidth]{IMG/comparativa cmm -cmm foto 10.png}
\caption{Zoom de las primeras posiciones}
\label{fig:Zoom de las primeras posiciones}
\end{figure}

\\



\subsection{Escalando Posiciones}

Una de las consignas del trabajo era encontrar una tecnica para hacer escalar en el ranking a un \textbf{equipo}, para lograr esto tenemos dos alternativas. \\


\subsubsection{El torneo ya finalizo}

En este escenario el torneo se encuentra finalizado y los resultados no pueden modificarse. Lo que proponemos es ver si modificando el orden de los partidos podemos influir en el ranking de un equipo. Para esto vamos a modificar el orden de sus victorias de forma tal de encontrar una que resulte en una mejoría de su ranking. \\


\subsubsection{Agregando partidos}

En este caso vamoa a analizar si podemos influir positivamente en el ranking a favor de un equipo, minimizando la cantidad de partidos ganados. \\

Nuestra teoría es que tomando el conjunto de equipos que perdio contra el seleccionado y haciendolos jugar y ganar a los principales del ranking, vamos a lograr que nuestro equipo mejore en la tabla de posiciones. \\



\subsection{Análisis Cuantitativo}


Vamos a estudiar la eficiencia de ambas tecnicas incrementando y variando los volúmenes de datos. La idea es repetir el cómputo de los rankings para la misma instancia de datos al azar, y posteriormente ir incrementando la cantidad de datos. \\

Nuestra hipótesis es el que método de basado en \textbf{WP} va a tardar lo mismo para instancias de datos iguales, y se irá incrementando de forma casi lineal a medida que incrementemos los datos. En cambio con \textbf{CMM} basando en \textbf{Eliminación Gaussiana} y \textbf{Cholesky} esperamos que difieran en para las mismas intancias. Nuestra hipótesis sobre esto es que la implementación de \textbf{Cholesky} va a demorar menos tiempo. \\

Para esta prueba se generaron schedules variando la cantidad de equipos en 6, 50, 100, 200, 300, 500, 700, 1000 y 2000. \\

Adicionalmente se varió la cantidad de partidos jugados por cada equipo. Dado el análisis de complejidad de los algoritmos implementados, solo varían el tiempo de calculo en funcion del tamaño de la matriz definida por la cantidad de equipos, por lo cual al variar la cantidad de partidos no esperamos encontrarnos con variaciones en el tiempo. \\

Ejecutamos los test para nuestra implementación de Eliminación Gaussiana.A continuación se muestran los resultados de tiempos de ejecucion dependiendo la cantidad de equipos. Para evidenciar la complejidad cubica del algoritmo lo hemos encerrado entre 2 funciones cuadraticas que evidencian que no pueden contener la curva de tiempos de nuestro algoritmo. \\


\begin{figure}[H]
\centering
\includegraphics[width=1\textwidth]{IMG/gauss cuadrativo.png}
\caption{Gauss cuadrático}
\label{fig:Gauss cuadrático}
\end{figure}

\\

Luego observamos la misma gráfica pero con lineas de referencia de 2 funciones cubicas. \\

\begin{figure}[H]
\centering
\includegraphics[width=1\textwidth]{IMG/gauss cubico.png}
\caption{Gauss cúbico}
\label{fig:Gauss cúbico}
\end{figure}

\\
Ejecutamos los test para nuestra implementacion de Cholesky.A continuación se muestran los resultados de tiempos de ejecucion dependiendo la cantidad de equipos. Para mostrar la complejidad cúbica del algoritmo lo hemos encerrado entre 2 funciones cuadráticas que evidencian que no pueden contener la curva de tiempos de nuestro algoritmo.\\


\begin{figure}[H]
\centering
\includegraphics[width=1\textwidth]{IMG/cholesky cuadratico.png}
\caption{Cholesky cuadrático}
\label{fig:Cholesky cuadrático}
\end{figure}

\\

Luego observamos la misma grafica pero con lineas de referencia de 2 funciones cúbicas.\\

\begin{figure}[H]
\centering
\includegraphics[width=1\textwidth]{IMG/cholesky cubicos.png}
\caption{Cholesky cúbico}
\label{fig:Cholesky cúbico}
\end{figure}

\\

Ejecutamos los test para nuestra implementacion de WP. A continuación se muestran los resultados de tiempos de ejecucion dependiendo la cantidad de equipos: \\


\begin{figure}[H]
\centering
\includegraphics[width=1\textwidth]{IMG/wp lineal.png}
\caption{WP lineal}
\label{fig:WP lineal}
\end{figure}

\\


Cuando leimos el enunciado encontramos una frase que nos llamo la atención y era la siguiente: \\

"Se pide comparar, para distintos tamaños de matrices, el tiempo de computo requerido para cada metodo en el contexto donde la información de la matriz del sistema (C) se mantiene invariante, pero varia el termino independiente (b)"\\

Entendimos que nuestros tiempos de calculo no debian variar con la modificacion del termino independiente, pero para resolver esta incognita decidimos realizar la prueba.

Lo que hicimos fue tomar una matriz C, calcularle CMM, luego modificar algunos partidos de la matriz y cambiarlos (esto significa cambiar el resultado de ganados/perdidos) y calculamos nuevamente CMM. \\

Nuestra experimentación intenta demostrar que el tiempo de cálculo no cambia, aunque se varie el termino independiente.
A continuación se grafican ambos tiempos.\\

\begin{figure}[H]
\centering
\includegraphics[width=1\textwidth]{IMG/Cholesky con otro termino independiente.png}
\caption{Cholesky con otro termino independiente}
\label{fig:Cholesky con otro termino independiente}
\end{figure}



\pagebreak
\section{Discusión}
En esta sección presentamos nuestras conclusiones sobre los resultados obtenidos de los experimentos del punto anterior. \\

\subsection{Ranking}

De los resultados obtenidos podemos ver que el ranking obtenido con \textbf{WP} no es muy realista, ya que la primer posición es ocupada por un participante que jugo y gano un solo partido. \\

El ranking obtenido por \textbf{CMM} refleja de forma mucho m\'as realista el desempeño de cada jugador en el torneo. \\

En un escenario donde tenemos participantes que jugaron una cantidad distinta de partidos pensamos que refleja mejor la realidad del torneo el metodo de \textbf{CMM}. \\


\subsection{\¿Importa contra quien se pierde\?}

Como podemos observar realmente importa contra quien se pierde, del experimento realizado observamos que perder contra el participante último afecta m\'as el puntaje del ranking que perdiendo 
contra el primero. \\

La hipótesis con la que calculamos el experimento resulto ser falsa. Analizando más ejecuci\'ones llegamos a la conclusión de que lo resultados obtenidos son lógicos, ya que con esta técnica 
es m\'as esperable que un equipo de mitad de tabla tenga un resultado adverso contra los primeros, por lo cual la perdida de ranking es menor. \\


\subsection{Racha ganadora}

Por lo visto el jugador escalo rapidamente desde el último puesto en la tabla de posici\'ones a casi la mitad de tabla. Si bien mejoro su posición, no alcanzo el top ten, y en las últim\'as victorias su ascenso fue m\'as lento. Esto nos hace concluir que solo haciendo jugar y ganar a un participante, la capacidad que tiene para crecer en el ranking esta limitada por la falta de juego de sus rivales.

\\
Además sus victorias representaron una mejoría en el ranking de los participantes que lo vencieron a el. De esta forma podemos comprobar que la racha de un jugador afecta al ranking global.\\


\subsection{Escalando Posici\'ones}
Como mensionamos al principio y confirmamos con estos experimentos es que si importa a quien hay que ganarle ya que como se demostro no es lo mismo ganarle al inmediato en el 
ranking que al que se encuentra primero,
de esta manera la forma m\'as rapida de llegar al primer puesto es ganandole siempre al que se encuentre primero.

\subsection{Análisis Cuantitativo}

Como era de esperar en el caso de \textbf{WP} para instancias el tiempo de ejecución fue el mismo, y el tiempo demorado a medida que crecian los datos de la instancia fue l\'ineal.

En cambio en el caso de \textbf{CMM} la implementación de \textbf{Cholesky} fue m\'as eficiente para las mism\'as instancias, y relativamente mejor a medida que se incrementaban los datos. 
Esto es esperado ya que nuestras implementaci\'ones se basaron en las propuestas por el libro \textbf{Burden}, que afirma que \textbf{Cholesky} consume 1/3 $n^3$ flops y 
\textbf{Eliminación Gaussiana} 2\/3 $n^3$ flops.


\subsection{La aritmética importa}

De los experimentos realizados notamos que es importante el tipo de datos utilizados. Principalmente cuando se utiliza \textbf{CMM}.
\\

Los errores de redondeo pueden derivar en un mal cálculo del ranking. Es decir, no considerar los suficientes decimales puede derivar en que un participante con un ranking
decimalmente menor quede mejor rankeado que otro con mayor puntaje.\\

Por ejemplo: El participante A con ranking 0,5819 y el participante B con ranking 0,5816 si se consideran solo dos decimales ambos tienen 0,58 y esto podria afectar su orden 
en el ranking global. \\

Para evitar esta situaci\'on nuestra implementación usa el tipo de datos float con con 5 decimales despues de la coma.\\


\subsection{Empates}

Encontramos que los empates pueden modelarse en el caso de \textbf{WP}, asignando un puntaje al partido empatado y continuando con el procedimiento normal.

En el caso de \textbf{CMM} nos resulto muy díficil tratar de modelarlo, como alternativa a este resultado proponemos modelarlo como si ambos equipos perdieran. Esto nos permite 
reutilizar el método y de alguna forma penar a ambos equipos por no haber ganado su partido.


















\pagebreak
\section{Conclusiones}

\subsection{Conclusiones}


Luego de la experimentación y análisis de los resultados, concluimos el método de calculo basado en \textbf{CMM} es mas justo en el caso de torneos donde los equipos no juegan la misma cantidad de partidos y donde el empate no es una opción. Ya que asigna un puntaje en base no solo a los resultados obtenidos, sino contra quien fueron obtenidos. Obteniendo un ranking basado en la meritocracia del resultado. \\

Para el caso de torneos donde cada equipo juegue la misma cantidad de partidos el método de \textbf{WP} a nuestro criterio resulta mas justo. Debido a que todos se pusieron a prueba la misma cantidad de veces. \\

Respecto a que implementación de \textbf{CMM} resulta mas eficiente. La conclusion es que depende. Ambas obtienen el mismo resultado, la principal ventaja de \textbf{Cholesky} es que realiza menos computos, mientras que la de \textbf{Eliminación Gaussiana} es que es mas sencilla su implementación. \\

Por último sobre \textbf{La utilización de técnicas avanzadas de análisis de datos son imprescindibles para mejorar cualquier deporte}, consideramos que la frase no es del todo cierta. Afortunadamente la frialdad de los números no es aplicable a la pasión de todos los deportes. Mientras en contados deportes el resultado puede predecirse de antemano, debido a las caracteristicas de los rivales, como en el caso del Polo, esta analogía no puede aplicarse a deportes como el Fútbol donde en innumerables ocasiones el equipo menos favorito termina llevandose el partido.


\pagebreak
\section{Apéndice}

\subsection{Entrada y salida de los algoritmos}

Dados los requerimientos de la c\'atedra el programa toma como parametros 3 argumentos, el primero es el archivo de entrada, luego el archivo de salida y
por \'ultimo el modo.  La c\'atedra solicitaba 3 modos el modo 0,1 y 2 para los 3 m\'etodos solicitados para ejecutar sobre la matriz Colley, luego agregamos 3 modos m\'as utilizados durante la etapa 
de experimentaci\'on.
\\
\begin{enumerate}
    \item Eliminaci\'on Gaussiana(EG)
    \item Factorizaci\'on de Cholesky(CL)
    \item WP
    \item Cholesky con modificaci\'on de partidos jugados
    \item Cholesky haciendo ganar al \'ultimo con el siguiente reiteradas veces hasta quedar primero
    \item Cholesky haciendo ganar al \'ultimo con el primero del ranking hasta quedar primero
\end{enumerate}
En todos los modos como paso previo a la realizaci\'on de algunos de los m\'etodos arma la matriz de Colley, explicada en la secci\'on siguiente , para luego utilizando los m\'etodos 0-1 en el programa (1 o 2 en la lista) resolver el sistema pedido.
\\

El modo 3 corre cholesky, como paso siguiente busca 2 equipos que hayan jugado previamente para cambiar su resultado y luego volver a ejecutar cholesky,
Este modo contiene un ciclo para repetir esta operaci\'on 100 veces.
\\
Para el modo 4 corre cholesky y luego ejecuta un ciclo donde el objetivo es lograr que el que haya salido \'ultimo luego de obtener el ranking de cholesky llegue al primer puesto ganandole
al que tiene por arriba inmediato en el ranking. En cada paso agrega un partido m\'as y vuelve a calcular cholesky para la nueva matriz, asi obtenemos un nuevo ranking y continuamos iterando hasta quedar en la primera posici\'on,
siempre utilizando al que salio \'ultimo en la primera utilizaci\'on del metodo de cholesky.

Una vez finalizado por stdout devuelve la cantidad de partidos jugados, adem\'as de que en el archivo  rankingSTEPS\_4.out dentro de la carpeta test se encuentran los rankings en cada iteraci\'on y como es la evoluci\'on.
El modo 5 es similar al anterior con la sutil diferencia que el participante que salio \'ultimo la primera vez que corrio cholesky llegue al primer puesto jugandole al que se encuentra en el primer puesto en cada iteraci\'on.

\newline
Tanto el formato de entrada y de salida del programa son los solicitados por la c\'atedra.
Para todos los m\'etodos el archivo de entrada es el mismo, que contiene el siguiente formato:\\
\newline

\begin{pmatrix}
    (n) & (k) \\
    (x1) & (e1) & (r1) & (t1) & (s1)\\
    (x2) & (e2) & (r2) & (t2) & (s1) \\
    ...\\
    (xk) & (ek) & (rk) & (tk) & (s1)\\
\end{pmatrix}\\

\newline
La primer l\'inea tiene 2 valores $n$ representa la cantidad de equipos y $k$ representa la cantidad de partidos, seguido k l\'ineas que representa cada partido, donde
$x1$ representa una fecha que en nuestro caso no utilizamos, luego $ei$ y $ti$ representan los numeros de los equipos, y por \'ultimo $ri$ y $si$ representan las anotaci\'ones de cada equipo respectivamente.\\
en nuestra experimentaci\'on con fines de no complejizar a\'un m\'as el problema no utilizamos en campo que representa la fecha si no que asumimos que dado el orden que venian los resultados era el orden de los partidos.

Luego una vez se ejecutan los m\'etodos 0-4 (1 a 4 en la lista)  devuelven un archivo de n l\'ineas donde en la l\'inea se obtiene el ranking del equipo i.

Para los m\'etodos 4 y 5 se devuelve el ranking para cada iteraci\'on antes de jugar el partido, estas se repiten hasta que el que comenzo \'ultimo termine primero utilizando dos m\'etodos descriptos anteriormente, en la ultima l\'inea se obtiene la cantidad total de partidos,
estos dos m\'etodos adem\'as devuelven por stdout la cantidad de partidos jugados hasta que termino en la primer posici\'on el que comenzo \'ultimo.



\subsection{Archivos de test usados}
Dentro de la carpeta /src/tests se encuentran los siguientes archivos usados en la experimentaci\'on
\begin{itemize}
 \item ATP2007.in este archivo es usado en la experimentaci\'on de escalar posici\'ones gandandole al siguiente o al primero del ranking
 \item ATP2007\_100.in este archivo se utilizo en el analisis de salto de posici\'ones en cuanto a 100 partidos jugados para obtener una cota
 \item test1.in archivo provisto por la materia
 \item test2.ina archivo de prueba provisto por la materia
 \item carpeta random test tiene todos los archivos de test que se utilizaron para la medici\'on de tiempos
\end{itemize}



\end{document}
