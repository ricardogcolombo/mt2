\subsection{Discusión}


\subsubsection{Ranking}

De los resultados obtenidos podemos ver que el ranking obtenido con \textbf{WP} no es muy realista, ya que la primer posición es ocupada por un participante que jugo y gano un solo partido. \\

El ranking obtenido por \textbf{CMM} refleja de forma mucho mas realista el desempeño de cada jugador en el torneo. \\

En un escenario donde tenemos participantes que jugaron una cantidad distinta de partidos pensamos que refleja mejor la realidad del torneo el metodo de \textbf{CMM}. \\


\subsubsection{¿Importa contra quien se pierde?}

Como podemos observar realmente importa contra quien se pierde, del experimento realizado observamos que perder contra el participante último afecta mas el puntaje del ranking que perdiendo contra el primero. \\

La hipótesis con la que calculamos el experimento resulto ser falsa. Analizando más ejecuciones llegamos a la conclusión de que lo resultados obtenidos son lógicos, ya que con esta técnica es mas esperable que un equipo de mitad de tabla tenga un resultado adverso contra los primeros, por lo cual la perdida de ranking es menor. \\


\subsubsection{Racha ganadora}

Por lo visto el jugador escalo rapidamente en la tabla de posiciones, y ademas mejoro el ranking de los participantes que lo vencieron a el.\\

Si bien mejoro su posición en la tabla, no alcanzo el top ten, y en las últimas victorias su ascenso fue mas lento. Esto nos hace concluir que solo haciendo jugar y ganar a un participante, la capacidad que tiene para crecer en el ranking esta limitada por la falta de juego de sus rivales.



\subsubsection{Escalando Posiciones}

\subsubsubsection{El torneo ya finalizo}

Poner resultados


\subsubsubsection{Agregando partidos}

Poner resultados
