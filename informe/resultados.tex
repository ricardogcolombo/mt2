Para analizar la efectividad y ecuanimidad de esta nueva forma de calcular el ranking vamos a realizar una serie de test a fin de obtener un analisis cuantitativo y cualitativo que nos permita compararlo con el clásico método de \textbf{WP}. \\
Con los test esperamos encontrar ventajas y desventajas de esta forma de medición, particularmente en escenarios donde no todos los participantes juegen la misma cantidad de partidos.
\\

Además realizaremos una comparación de los métodos de \textbf{Eliminación Gaussiana} y \textbf{Cholesky} para ver cual de los dos computa los rankings de manera mas eficiente.
\\

En esta sección solo presentaremos los experimentos realizados y los resultados obtenidos. Las conclusiones de cada experimento
las presentaremos en la seguiente sección. 


\subsection{Ranking}

Vamos a comparar la tabla de ranking obtenida a partir de un set de datos de la \textbf{ATP 2007}. Es decir calculamos el ranking a partir de la técnica \textbf{WP}, considerando partidos ganado / partidos jugados, a pesar de que no todos los jugadores hayan participado de la misma cantidad de partidos. Comparandolo con el \textbf{CMM} implementado con \textbf{Eliminación Gaussiana} y \textbf{Cholesky}. \\


\begin{figure}[H]
\centering
\includegraphics[width=1\textwidth]{IMG/Comparativa WP- CMM todos.png}
\caption{Rankings Calculados con las 3 tecnicas}
\label{fig:Comparacion de tecnicas}
\end{figure}

\\

Hemos realizado un zoom dentro del grafico para corroborar el nombre y posicion de ambos rankings.
\\

\begin{figure}[H]
\centering
\includegraphics[width=1\textwidth]{IMG/comparativa WP - CMM zoom.png}
\caption{Zoom de las primeras posiciones}
\label{fig:Zoom de las primeras posiciones}
\end{figure}

\\

\subsection{¿Importa a quien se le gana?}


En el escenario que se utilize \textbf{WP} realmente no importa a que equipo se le gane, ya que todos los partidos tienen la misma importancia y se les asigna el mismo puntaje. Pero en el caso de \textbf{CMM} resulta mas interesante plantearse esta pregunta. \\

La hipótesis que tenemos es que tomando un equipo de mitad de tabla, que denominamos \textbf{medio} el hecho de que le gane al lider de la tabla va a mejorar mucho mas el ranking que derrotando al que ocupe la última posición. \\

Realizamos un test tomando al equipo \textbf{medio}, y agregando un partido victorioso contra el puntero y analizamos como se modifica su ranking. Luego tomamos la tabla inicial, es decir sin ganarle al puntero, y repetimos el experimento esta vez derrotando al último. \\

Presentamos los resultados obtenidos.

\\


******Aca van los graficos de: tabla inicial, perder contra el primero y perder contra el ultimo (sacando el partido con el primero)

\\




\subsection{¿Importa contra quien se pierde?}

Para verificar si importa contra que equipo se juega, proponemos el test de tomar un equipo que se encuentra por la mitad de la tabla. Por notación denominamos a este equipo como \textbf{medio}. \\

Para lograrlo calculamos un ranking a partir de un set de datos. Y luego generamos una nueva instancia enfrentando a \textbf{medio}contra el actual puntero y calculamos el nuevo ranking. Volvemos a tomar la primer instancia y lo enfrentamos contra el último, calculamos nuevamente el ranking y luego comparamos los tres rankings obtenidos. La premisa que tenemos del experimento es que el ranking del equipo \textbf{medio} no debería ser afectado por el rival contra el que perdió.\\

Este experimento lo calculamos usando la técnica de \textbf{CMM}, ya que considerar \textbf{WP} no afecta el resultado.

\\
A continuación presentamos los graficos obtenidos.

\\


******Aca van los graficos de: tabla inicial, perder contra el primero y perder contra el ultimo (sacando el partido con el primero)

\\



\subsection{Racha ganadora}

Para estudiar la ecuanimidad del \textbf{CMM} realizamos un experimento tomando al participante del \textbf{ATP 2007} que se encontraba en el último puesto y le asignamos una racha ganadora contra los primeros diez jugadores del ranking. \\

Además este test nos permite observar como la racha de un jugador afecta al ranking global y si ganandole a los mejores realmente escala una considerable cantidad de posiciones en el ranking. \\

A continuación presentamos el ranking calculado construido de la siguiente manera: \\

\begin{itemize}
	\item Eje X el valor obtenido al ejecutar CMM implementado con Cholesky.
	\item Eje Y el valor obtenido al ejecutar CMM implementado con Eliminación Gaussiana.
	\item El tamaño de la burbuja es la cantidad de partidos jugados.
\end{itemize}

\\

\begin{figure}[H]
\centering
\includegraphics[width=1\textwidth]{IMG/comparativa cmm -cmm foto 0.png}
\caption{Zoom de las primeras posiciones}
\label{fig:Zoom de las primeras posiciones}
\end{figure}

\\
Y a continuación el ranking después de los 10 partidos: \\
\\
En el gráfico se observa mediante una flecha cual es la ganancia en Ranking que tiene el ultimo jugador al ganarle a los top 10.\\

\begin{figure}[H]
\centering
\includegraphics[width=1\textwidth]{IMG/comparativa cmm -cmm foto 10.png}
\caption{Zoom de las primeras posiciones}
\label{fig:Zoom de las primeras posiciones}
\end{figure}

\\



\subsection{Escalando Posiciones}

Una de las consignas del trabajo era encontrar una tecnica para hacer escalar en el ranking a un \textbf{equipo}, para lograr esto tenemos dos alternativas. \\


\subsubsection{El torneo ya finalizo}

En este escenario el torneo se encuentra finalizado y los resultados no pueden modificarse. Lo que proponemos es ver si modificando el orden de los partidos podemos influir en el ranking de un equipo. Para esto vamos a modificar el orden de sus victorias de forma tal de encontrar una que resulte en una mejoría de su ranking. \\


\subsubsection{Agregando partidos}

En este caso vamoa a analizar si podemos influir positivamente en el ranking a favor de un equipo, minimizando la cantidad de partidos ganados. \\

Nuestra teoría es que tomando el conjunto de equipos que perdio contra el seleccionado y haciendolos jugar y ganar a los principales del ranking, vamos a lograr que nuestro equipo mejore en la tabla de posiciones. \\



\subsection{Análisis Cuantitativo}


Vamos a estudiar la eficiencia de ambas tecnicas incrementando y variando los volúmenes de datos. La idea es repetir el cómputo de los rankings para la misma instancia de datos al azar, y posteriormente ir incrementando la cantidad de datos. \\

Nuestra hipótesis es el que método de basado en \textbf{WP} va a tardar lo mismo para instancias de datos iguales, y se irá incrementando de forma casi lineal a medida que incrementemos los datos. En cambio con \textbf{CMM} basando en \textbf{Eliminación Gaussiana} y \textbf{Cholesky} esperamos que difieran en para las mismas intancias. Nuestra hipótesis sobre esto es que la implementación de \textbf{Cholesky} va a demorar menos tiempo. \\

Para esta prueba se generaron schedules variando la cantidad de equipos en 6, 50, 100, 200, 300, 500, 700, 1000 y 2000. \\

Adicionalmente se varió la cantidad de partidos jugados por cada equipo. Dado el análisis de complejidad de los algoritmos implementados, solo varían el tiempo de calculo en funcion del tamaño de la matriz definida por la cantidad de equipos, por lo cual al variar la cantidad de partidos no esperamos encontrarnos con variaciones en el tiempo. \\

Ejecutamos los test para nuestra implementación de Eliminación Gaussiana.A continuación se muestran los resultados de tiempos de ejecucion dependiendo la cantidad de equipos. Para evidenciar la complejidad cubica del algoritmo lo hemos encerrado entre 2 funciones cuadraticas que evidencian que no pueden contener la curva de tiempos de nuestro algoritmo. \\


\begin{figure}[H]
\centering
\includegraphics[width=1\textwidth]{IMG/gauss cuadrativo.png}
\caption{Gauss cuadrático}
\label{fig:Gauss cuadrático}
\end{figure}

\\

Luego observamos la misma gráfica pero con lineas de referencia de 2 funciones cubicas. \\

\begin{figure}[H]
\centering
\includegraphics[width=1\textwidth]{IMG/gauss cubico.png}
\caption{Gauss cúbico}
\label{fig:Gauss cúbico}
\end{figure}

\\
Ejecutamos los test para nuestra implementacion de Cholesky.A continuación se muestran los resultados de tiempos de ejecucion dependiendo la cantidad de equipos. Para mostrar la complejidad cúbica del algoritmo lo hemos encerrado entre 2 funciones cuadráticas que evidencian que no pueden contener la curva de tiempos de nuestro algoritmo.\\


\begin{figure}[H]
\centering
\includegraphics[width=1\textwidth]{IMG/cholesky cuadratico.png}
\caption{Cholesky cuadrático}
\label{fig:Cholesky cuadrático}
\end{figure}

\\

Luego observamos la misma grafica pero con lineas de referencia de 2 funciones cúbicas.\\

\begin{figure}[H]
\centering
\includegraphics[width=1\textwidth]{IMG/cholesky cubicos.png}
\caption{Cholesky cúbico}
\label{fig:Cholesky cúbico}
\end{figure}

\\

Ejecutamos los test para nuestra implementacion de WP. A continuación se muestran los resultados de tiempos de ejecucion dependiendo la cantidad de equipos: \\


\begin{figure}[H]
\centering
\includegraphics[width=1\textwidth]{IMG/wp lineal.png}
\caption{WP lineal}
\label{fig:WP lineal}
\end{figure}

\\


Cuando leimos el enunciado encontramos una frase que nos llamo la atención y era la siguiente: \\

"Se pide comparar, para distintos tamaños de matrices, el tiempo de computo requerido para cada metodo en el contexto donde la información de la matriz del sistema (C) se mantiene invariante, pero varia el termino independiente (b)"\\

Entendimos que nuestros tiempos de calculo no debian variar con la modificacion del termino independiente, pero para resolver esta incognita decidimos realizar la prueba.

Lo que hicimos fue tomar una matriz C, calcularle CMM, luego modificar algunos partidos de la matriz y cambiarlos (esto significa cambiar el resultado de ganados/perdidos) y calculamos nuevamente CMM. \\

Nuestra experimentación intenta demostrar que el tiempo de cálculo no cambia, aunque se varie el termino independiente.
A continuación se grafican ambos tiempos.\\

\begin{figure}[H]
\centering
\includegraphics[width=1\textwidth]{IMG/Cholesky con otro termino independiente.png}
\caption{Cholesky con otro termino independiente}
\label{fig:Cholesky con otro termino independiente}
\end{figure}
