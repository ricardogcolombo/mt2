En este trabajo practico intentaremos modelar y resolver el problema de generar un ranking de equipos a partir de los resultados entre ellos con la condicion de que no haya empates entre ellos. 
Para confenccionar dicho ranking haremos uso de 2 metodos diferentes.El primero
es el Winning Porcentage y el 2 es el Colley Matrix Method (CMM).
El WP es simplemente Partidos Ganados / Partidos Jugados mientras que el CMM
requiere mas explicacion.\newline
Sea T = $\{$1,2...n$\}$ el conjunto de los equipos.Dado un i $\in$ T definimos:\newline
$n_i$ a la cantidad de partidos jugados del equipo i
$w_i$ a la cantidad de partidos ganados del equipo i
$l_i$ a la cantidad de partidos perdidos del equipo i
Dados i y j $\in$ T llamarems $n_{ij}$ a la cantidad de partidos jugados entre ellos . Notar
que $n_{ij}$ es igual $n_{ji}$
blah blah 

Esto nos lleva a un sistema de la forma Cr = b con C $\in$ R

\newpage
\subsection{Entrada y salida de los algoritmos}

Dados los requirimientos de la catedra el programa toma como parametros 3 argumentos.
El primero de ellos es la ubicacion del archivo con el historial de las partidos respetando
el formato descripto (en algun lado ) . El segundo es la salida del ranking correspodiente y el ultimo indica que metodo utilizar para calcular el ranking.
