Las Competencias deportivas , de cualquier indole requieren la comparacion de equipos
mediante la confeccion de las tablas de Posiciones y rankings en base a los resultados obtenidos
durante un cierto periodo de tiempo.


En este trabajo practico intentaremos modelar y 
resolver el problema de generar un ranking de equipos a partir de los 
resultados con la condicilon de que no haya empates. 
Para confenccionar dicho ranking haremos uso de 2 metodos diferentes.
El primero es el Winning Porcentage y el 2 es el Colley Matrix Method (CMM).
Con estos metodos es posible obtener 2 rankings y nos proponemos hallar
la similitudes y diferencia entre ellos.

¿Por que es importante el ranking ?

Es importante porque determina quienes fueron los mejores y peores equipos al final
de la temporada , quienes avanzan a la siguiente ronda y en ciertas
ligas como la liga de basquet y de baseball de los estados unidos 
determina la prioridad para los drafts.
Ademas de eso tenemos la justicia deportiva y una gran cantidad de dinero en juego

¿Por que es importante el analisis de datos ?

Es importanta porque permite en las manos adecuadas, optimizar el
rendimiento de los jugadores y ademas decidir si los jugadores son buenos en base a ciertos 
indicadores.



%En este trabajo practico intentaremos modelar y resolver el problema de generar un ranking de equipos a partir de los resultados entre ellos con la condicion de que no haya empates entre ellos. 
%Para confenccionar dicho ranking haremos uso de 2 metodos diferentes.El primero
%es el Winning Porcentage y el 2 es el Colley Matrix Method (CMM).
%El WP es simplemente Partidos Ganados / Partidos Jugados mientras que el CMM
%requiere mas explicacion.\newline
%Sea T = $\{$1,2...n$\}$ el conjunto de los equipos.Dado un i $\in$ T definimos:\newline
%$n_i$ a la cantidad de partidos jugados del equipo i
%$w_i$ a la cantidad de partidos ganados del equipo i
%$l_i$ a la cantidad de partidos perdidos del equipo i
%Dados i y j $\in$ T llamarems $n_{ij}$ a la cantidad de partidos jugados entre ellos . Notar
%que $n_{ij}$ es igual $n_{ji}$
%blah blah 

%Esto nos lleva a un sistema de la forma Cr = b con C $\in$ R



