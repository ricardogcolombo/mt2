Las Competencias deportivas, de cualquier índole, requieren la comparación de equipos mediante la confección de tablas de Posiciones y ranking en base a los 
resultados obtenidos durante un cierto periodo de tiempo. Algunos de los métodos de ordenamiento utilizados normalmente se basan en proporción de victorias 
sobre partidos jugados, aunque no todos, sin embargo estos no logran mostrar las dificultades presentadas en estas competencias. A su vez este tipo de 
rankings son utilizados para clasificaciones de otras competencias o su fin dentro de las categorías en las cuales se encuentran divididas estas 
competencias, teniendo impactos económicos en los clubes, demostrando que no son del todo justas estos métodos utilizados.
En este trabajo practico intentaremos modelar y resolver el problema de generar un ranking de equipos a partir de los resultados exceptuando en un principio los empates,
luego discutiremos como podríamos modelar esto.
Para confeccionar dicho ranking presentaremos los siguientes métodos:
\begin{itemize}  
\item El Porcentaje de Victorias \textbf{WP} : Este método básicamente es el que nombramos anteriormente, se basa en armar el ranking en base al porcentaje de victorias sobre partidos totales.
\item Colley Matrix Method \textbf{CMM} :Este es un método matricial se basa en el armado de un ranking en base a los partidos ganados, perdidos y los partidos totales entre equipos.
\end{itemize}

Como mencionamos anteriormente el método de Colley, es un método matricial, por lo que para la resolución de este sistema utilizaremos los métodos vistos en
la catedra para encontrar los valores del ranking a la solucion Ax=b dichos métodos son:

\begin{itemize}  
\item La Eliminación Gaussiana \textbf{EG}
\item La Factorización de Cholesky \textbf{CL}
\end{itemize}

Luego se procederá a un análisis de los distintos métodos de resolución de sistemas lineales, EG y CL, en base a tiempos y tamaños de sistema a resolver,
por otro lado estaremos analizando los resultados obtenidos del método de Colley, haremos una comparativa contra el método de porcentajes de victorias para 
finalizar realizando algún tipo de conclusión respecto a la justicia de dichos métodos frente a los resultados. Todos estos resultados serán representados 
mediante gráficos comparativos con sus respectivos detalles.


