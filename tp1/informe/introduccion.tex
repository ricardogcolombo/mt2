\subsection{Motivacion}

Las Competencias deportivas, de cualquier \'\i{ndole} requieren la comparación de equipos
mediante la confección de las tablas de Posiciones y rankings en base a los resultados obtenidos
durante un cierto periodo de tiempo.
En un mundo ideal, cada equipo jugar\'\i{a} exactamente la misma cantidad de partidos y se enfrentar\'\i{a} a sus rivales
la misma cantidad de veces. Esto no siempre pasa, puede haber demasiados equipos o bien pueden ser competencias
eliminatorias, donde el que pierde deja de competir. En esos casos hace faltar tener otras técnicas  

\newline
\subsection{Problema a Resolver}
En este trabajo practico intentaremos modelar y 
resolver el problema de generar un ranking de equipos a partir de los 
resultados con la condición de que no haya empates. 
Para confeccionar dicho ranking presentaremos los siguientes métodos
\newline
- El Porcentaje de Victorias (En adelante llamado WP)
\newline
- Colley Matrix Method (En adelante llamado CMM)


Además, para resolver el CMM presentaremos y usaremos las siguientes técnicas.\\
-La Eliminacion Gaussiana (En adelante llamado EG)\\
-La Factorizacion de Cholesky (En adelante llamado CL)  


\subsection{Objetivos}
-Hacer un análisis de WP y CMM, hallar los puntos fuertes y débiles de cada método y compararlos 
-Comparar EG con CL. Ver cual es mas rápido. 

\\
\subsection{¿Por que es importante el ranking?} 
\\
Es importante porque determina quienes fueron los mejores y peores equipos al final
de la temporada, quienes avanzan a la siguiente ronda y en ciertas
ligas como la liga de basquet y de baseball de los estados unidos 
determina la prioridad para los drafts.
Además de eso tenemos la justicia deportiva y una gran cantidad de dinero en juego

\subsection{¿Por que es importante el análisis de datos ?} 

Es importante porque permite en las manos adecuadas, optimizar el
rendimiento de los jugadores y además decidir si los jugadores son buenos en base a ciertos 
indicadores.


%En este trabajo practico intentaremos modelar y resolver el problema de generar un ranking de equipos a partir de los resultados entre ellos con la condicion de que no haya empates entre ellos. 
%Para confenccionar dicho ranking haremos uso de 2 metodos diferentes.El primero
%es el Winning Porcentage y el 2 es el Colley Matrix Method (CMM).
%El WP es simplemente Partidos Ganados / Partidos Jugados mientras que el CMM
%requiere mas explicacion.\newline
%Sea T = $\{$1,2...n$\}$ el conjunto de los equipos.Dado un i $\in$ T definimos:\newline
%$n_i$ a la cantidad de partidos jugados del equipo i
%$w_i$ a la cantidad de partidos ganados del equipo i
%$l_i$ a la cantidad de partidos perdidos del equipo i
%Dados i y j $\in$ T llamarems $n_{ij}$ a la cantidad de partidos jugados entre ellos . Notar
%que $n_{ij}$ es igual $n_{ji}$
%blah blah 

%Esto nos lleva a un sistema de la forma Cr = b con C $\in$ R



