

En las Competencias deportivas de cualquier índole, siempre nos encontramos comparando equipos o atletas a través de ciertos tipos de Rankings o Tablas de Posición. Los mismos utilizan diferentes criterios de evaluación y además suelen tener al tiempo como un factor vinculante. Un ejemplo de dicho vinculo sería el Torneo Clausura de Fútbol Argentino 2012 (10 de febrero - 24 de junio); el mismo tuvo una tabla posición que plasmó la calidad de los equipos con una validez igual a la duración del torneo.\\
Como ya comentábamos, existen diferentes métodos para posicionar a los deportistas o equipos; siendo uno de los más sencillos el de proporción de victorias sobre partidos jugados. Aunque efectivo, dicho método no logra reflejar la complejidad de las competencias dado que deja muchos detalles y características de las mismas fuera. Por ello diremos que es útil hasta cierto punto y esto es importante de tener en cuenta debido al impacto en el prestigio y la economía del atleta o club que queda muy abajo en la tabla de posición. En el caso del fútbol argentino, los equipos pueden descender de categorías y en consecuencia recibir menos dinero por la participación en los torneos de menor categoría (caer de 1ra A Nacional a 1ra B Nacional) - este impacto económico puede ser muy dañino para una institución.\\

El foco de este trabajo practico es el de modelar y resolver la problemática de generar un ranking de equipos a partir de los resultados, exceptuando los empates. Posteriormente discutiremos como podríamos incluirlos.
Para la confección de dicho ranking presentaremos los siguientes métodos:
\begin{itemize}  
\item Porcentaje de Victorias \textbf{WP} : Nombrado anteriormente, se basa calcular porcentajes basado en Victorias sobre Partidos Totales.
\item Colley Matrix Method \textbf{CMM} : Método matricial basado en los partidos ganados, perdidos y partidos totales entre equipos.
\end{itemize}

Dado que se trabaja con matrices en el método de Colley, utilizaremos diferentes técnicas (vistas en la catedra) para resolver los sistemas matriciales y así dar con los valores del ranking a la solución Ax=b.  Los mismos son:

\begin{itemize}  
\item La Eliminación Gaussiana \textbf{EG}
\item La Factorización de Cholesky \textbf{CL}
\end{itemize}

Por otro lado, se procederá a un análisis de los distintos métodos de resolución de sistemas lineales (EG y CL) en base a tiempos y tamaños de sistema a resolver. Adicionalmente, estaremos analizando los resultados obtenidos del método de Colley y haremos una comparativa contra el método de porcentajes de victorias. 
Finalmente, llegaremos a una conclusión respecto de la justicia de dichos métodos frente a los resultados. Cabe aclarar que todos estos resultados serán representados mediante gráficos comparativos con sus respectivos detalles.

