\subsection{Entrada y salida de los algoritmos}

Dados los requerimientos de la c\'atedra el programa toma como parametros 3 argumentos, el primero es el archivo de entrada, luego el archivo de salida y
por \'ultimo el modo.  La c\'atedra solicitaba 3 modos el modo 0,1 y 2 para los 3 m\'etodos solicitados para ejecutar sobre la matriz Colley, luego agregamos 3 modos m\'as utilizados durante la etapa 
de experimentaci\'on.
\\
\begin{enumerate}
    \item Eliminaci\'on Gaussiana(EG)
    \item Factorizaci\'on de Cholesky(CL)
    \item WP
    \item Cholesky con modificaci\'on de partidos jugados
    \item Cholesky haciendo ganar al \'ultimo con el siguiente reiteradas veces hasta quedar primero
    \item Cholesky haciendo ganar al \'ultimo con el primero del ranking hasta quedar primero
\end{enumerate}
En todos los modos como paso previo a la realizaci\'on de algunos de los m\'etodos arma la matriz de Colley, explicada en la secci\'on siguiente , para luego utilizando los m\'etodos 0-1 en el programa (1 o 2 en la lista) resolver el sistema pedido.
\\

El modo 3 corre cholesky, como paso siguiente busca 2 equipos que hayan jugado previamente para cambiar su resultado y luego volver a ejecutar cholesky,
Este modo contiene un ciclo para repetir esta operaci\'on 100 veces.
\\
Para el modo 4 corre cholesky y luego ejecuta un ciclo donde el objetivo es lograr que el que haya salido \'ultimo luego de obtener el ranking de cholesky llegue al primer puesto ganandole
al que tiene por arriba inmediato en el ranking. En cada paso agrega un partido m\'as y vuelve a calcular cholesky para la nueva matriz, asi obtenemos un nuevo ranking y continuamos iterando hasta quedar en la primera posici\'on,
siempre utilizando al que salio \'ultimo en la primera utilizaci\'on del metodo de cholesky.

Una vez finalizado por stdout devuelve la cantidad de partidos jugados, adem\'as de que en el archivo  rankingSTEPS\_4.out dentro de la carpeta test se encuentran los rankings en cada iteraci\'on y como es la evoluci\'on.
El modo 5 es similar al anterior con la sutil diferencia que el participante que salio \'ultimo la primera vez que corrio cholesky llegue al primer puesto jugandole al que se encuentra en el primer puesto en cada iteraci\'on.

\newline
Tanto el formato de entrada y de salida del programa son los solicitados por la c\'atedra.
Para todos los m\'etodos el archivo de entrada es el mismo, que contiene el siguiente formato:\\
\newline

\begin{pmatrix}
    (n) & (k) \\
    (x1) & (e1) & (r1) & (t1) & (s1)\\
    (x2) & (e2) & (r2) & (t2) & (s1) \\
    ...\\
    (xk) & (ek) & (rk) & (tk) & (s1)\\
\end{pmatrix}\\

\newline
La primer l\'inea tiene 2 valores $n$ representa la cantidad de equipos y $k$ representa la cantidad de partidos, seguido k l\'ineas que representa cada partido, donde
$x1$ representa una fecha que en nuestro caso no utilizamos, luego $ei$ y $ti$ representan los numeros de los equipos, y por \'ultimo $ri$ y $si$ representan las anotaci\'ones de cada equipo respectivamente.\\
en nuestra experimentaci\'on con fines de no complejizar a\'un m\'as el problema no utilizamos en campo que representa la fecha si no que asumimos que dado el orden que venian los resultados era el orden de los partidos.

Luego una vez se ejecutan los m\'etodos 0-4 (1 a 4 en la lista)  devuelven un archivo de n l\'ineas donde en la l\'inea se obtiene el ranking del equipo i.

Para los m\'etodos 4 y 5 se devuelve el ranking para cada iteraci\'on antes de jugar el partido, estas se repiten hasta que el que comenzo \'ultimo termine primero utilizando dos m\'etodos descriptos anteriormente, en la ultima l\'inea se obtiene la cantidad total de partidos,
estos dos m\'etodos adem\'as devuelven por stdout la cantidad de partidos jugados hasta que termino en la primer posici\'on el que comenzo \'ultimo.



\subsection{Archivos de test usados}
Dentro de la carpeta /src/tests se encuentran los siguientes archivos usados en la experimentaci\'on
\begin{itemize}
 \item ATP2007.in este archivo es usado en la experimentaci\'on de escalar posici\'ones gandandole al siguiente o al primero del ranking
 \item ATP2007\_100.in este archivo se utilizo en el analisis de salto de posici\'ones en cuanto a 100 partidos jugados para obtener una cota
 \item test1.in archivo provisto por la materia
 \item test2.ina archivo de prueba provisto por la materia
 \item carpeta random test tiene todos los archivos de test que se utilizaron para la medici\'on de tiempos
\end{itemize}

