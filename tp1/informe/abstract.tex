\begin{abstract}
En el mundo de las competencias deportivas, existen numerosos rankings por los cuales se miden cuando un equpo es mejor que otro en la misma, analizaremos el método para la obtención de rankings mediante porcentaje de partidos ganados, y el método de Colley se basa en principios de probabilidad, utilizando solamente cantidad de partidos ganados, perdidos y totales como entrada para el desarrollo del ranking. Implementamos estos métodos, luego resolvemos la matriz de Colley con los algoritmos de resolución de ecuaciones lineales, Eliminación Gaussiana y Cholesky. & Equipo & Ranking \\ \hline
Comparamos ambos métodos de ordenamiento y obtuvimos distintas métricas para determinar cual parece ser más justo a la hora de decidir rankings.
\end{abstract}
