\begin{abstract}
A diferencia del método para la obtención de rankings mediante porcentaje de partidos ganados, el método de Colley se basa en principios de probabilidad, utilizando solamente cantidad de partidos ganados, perdidos y totales como entrada 
para el desarrollo de un ranking. Implementamos este método, luego lo resolvemos con los métodos de resolución de ecuaciones lineales, Eliminación Gaussiana y Cholesky.
Comparamos ambos métodos de ordenamiento y obtuvimos distintas métricas para determinar cual parece ser más justo a la hora de decidir rankings.
\end{abstract}
