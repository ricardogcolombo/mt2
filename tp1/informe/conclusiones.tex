Consideramos que ambos criterios se acercan a la realidad, a pesar de que no modelan los empates, de igual manera, nos inclinamos hacia el uso de $WP$ por sobre el método de Colley debido a la observación de que el encuentro de dos equipos modifica a un tercero.

Respecto a que implementación, a pesar de no haber usado la matriz factorizada, es más performante el método de Cholesky y nos inclinaríamos hacia el uso de este algoritmo en caso de ser necesario por sobre la Eliminación Gaussiana, siempre y cuando se cumplan las condiciones necesarias para la solución del sistema.

En contados deportes el resultado puede predecirse de antemano, debido a las características de los rivales, como en el caso del Polo. Caso Contrario, no puede aplicarse a deportes como el Fútbol, donde en innumerables ocasiones el equipo menos favorito termina llevándose el partido.\\
Hay casos en donde Sports Analitycs parece haber dado buenos resultados, sin ir m\'as lejos en el último mundial de futbol donde gano Alemania, se utilizó Sports Analistycs con una herramienta llamada SAP Sports One con la cual se pueden saber todas las estadísticas de los jugadores y se podrían tener datos de los rivales para realizar cierto análisis previo a cada partido, para este dejamos una nota de un diario canadiense en la bibliografía (item 3 bibliografía).

Por último sobre la frase \textbf{La utilización de técnicas avanzadas de análisis de datos son imprescindibles para mejorar cualquier deporte}, y en base a las películas vistas , consideramos que el Análisis de datos es una herramienta importante para mejorar el rendimiento deportivo de un equipo, siempre y cuando se utilicen correctamente. Las estadísticas son indicadores de que tan bien rinden los jugadores pero hay factores externos como el ánimo, el nivel de los competidores que no son considerados en las estadísticas ya que no son medibles y pueden modificar bastante el rendimiento de los jugadores.Afortunadamente la frialdad de los números no es aplicable a la pasión de todos los deportes.
