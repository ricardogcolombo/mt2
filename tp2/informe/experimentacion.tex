Para analizar los algortimos implementados vamos a utilizar los archivos test1.in y test2.in provistos por la cátedra y realizar una serie de test que nos permitirán en primer caso encontrar parametros buenos con lo que ejecutarlos y posteriormente evaluar el desempeño de la implementación mediante las métricas propuestas por la cátedra.

Dividimos la experimentación en dos secciones, una para cada archivo de prueba. Y evaluaremos los parametros individualmente para cada una de ellas


\subsubsubsection {Algoritmo de K-NN}

Lo primero que vamos a hacer es encontrar un valor de $K$ que nos permita maximizar la cantidad de aciertos, sin tener en consideración las métricas.

Para el archivo de test Test1.in, provisto por la cátedra. Ejecutamos el algoritmos de $K-NN$ variando los valores de k entre {1..20} dejando fija la cantidad de particiones en 10. Luego para cada K nos quedamos con los resultados de la iteración con mas aciertos.

Expresamos los aciertos para cada K en con el siguiente gráfico:

\begin{figure}
  \centering
  \includegraphics[width=0.7\columnwidth]{test1 - cantidadAciertos.bmp}
  \caption{Se presenta la cantidad de aciertos dependiendo de los K vecinos}
  \label{fig:test1 - cantidadAciertos.bmp}
\end{figure}

Como se puede observar la iteración que mas aciertos dió es para $K = 3$. Además se puede observar que a medida que se incrementa el valor de $K$ la cantidad de aciertos va disminuyendo levemente, cumpliendo lo mencionado en la introducción teórica. Cuanto mas corta sea la distancia de los vecinos, mas chances hay de tener un acierto.

Repetimos el experimento para el archivo Test2.in

\begin{figure}
  \centering
  \includegraphics[width=0.7\columnwidth]{test2 - cantidadAciertos.bmp}
  \caption{Se presenta la cantidad de aciertos dependiendo de los K vecinos}
  \label{fig:test2 - cantidadAciertos.bmp}
\end{figure}


Durante la ejecución de los test notamos que el tiempo que le tomaba al algoritmo encontrar una respuesta es constante, es decir la elección de los $K$ no influye en el tiempo de cómputo. Por lo que no consideramos necesario tomar métricas de medición del tiempo.

Caracterizados los valores de $K$ que nos dieron mejor cantidad de aciertos. Presentamos las métricas para el mejor caso de cada una de ellas.

$Presición$

$Recall$

$F1-Score$

Aca iria en forma de tabla y su explicacion


\subsubsubsection {Algoritmo de K-NN con Optimización de PCA}

\subsubsubsection {Algoritmo de K-NN con Optimización de PSL-DA}



