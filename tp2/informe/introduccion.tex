El objetivo de este trabajo práctico es desarollarar un clasificador que permita reconocer dígitos manuscritos.
Par llevarlo a cabo analizaremos técnicas de reconocimiento óptico de caracteres (OCR). Es un proceso dirigido a la digitalización de caracteres manuscritos, los cuales identifican automáticamente a partir de una imagen símbolos o caracteres que pertenecen a un determinado alfabeto, para luego almacenarlos en forma de datos.

Exploraremos la técnica de reconocimiento K-NN, K vecinos más próximos, además experimentaremos con distintas variantes de clasificación:

\begin{itemize}
\item PCA
\item PLSDA
\end{itemize}


Luego analizaremos la performance de cada una de estas técnicas mediante un set de experimentos, para evaluar las fortalezas y deficiencias de cada una de ellas. \\


\subsection{Introducción Teórica}







