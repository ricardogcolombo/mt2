% TO DO : PASAR A WORD ; ORTOGRAFIA Y CAMBIAR UN PAR DE COSAS
En este trabajo practico , haremos una primera incursion en el tema de Machine Learning mediante la implementacion de tecnicas para el reconocimiento de digitos.

Tenemos 2 Base de datos , una con los elementos ya identificados llamada train y otra con los elementos a etiquetar llamada test . Usando a train para preparar al algoritmo , se busca poder identificar a los elementos de test con la mayor eficacia posible .

La base de esto es el metodo conocido como kNN , que de manera resumida ,
considera a cada elemento ya identificado como un punto en el espacio y cada vez que llega un elementar sin identificar , se llama a elecciones con los elementos mas cercano y gana quien obtenga mas votos.
El principal problemita de este metodo es que su tiempo de ejecucion depende de la dimesion de los elementos y como nuestras imagenes son de 784 dimesiones ... bueno no quiero dejar la notebook un par de  semanas corriendo el test .
Para contrarestar esto , se implemeta los metodos conocidos como PCA 
y PLS-DA que basicamente hacen una transformacion sobre los datos de entrada y permiten reducir la dimesion de los elementos (WIII) , lo 
cual es muy bueno cuando la dimesion de los elementos es un problema



