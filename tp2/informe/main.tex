\documentclass[a4,11pt]{article}

\parindent=10pt
\parskip=6pt
%\usepackage[width=15.5cm, left=2.5cm, top=2cm, height= 24.5cm]{geometry}

\usepackage[paper=a4paper, left=2cm, right=2cm, bottom=2.5cm,top=2.5cm]{geometry}

% Paquetes de nacionalización. No olvidar para poder poner tildes!
\usepackage[spanish]{babel}
\usepackage[utf8]{inputenc}

% Paquetes para graficos
\usepackage{subfig}
% \usepackage{graphicx} %% La caratula lo incluye

% Paquetes para matematica
\usepackage{amsmath}
\usepackage{amsfonts}
\usepackage{amssymb}
% esto es para el codigo
\usepackage{listings}
% Paquetes para pseudo
\usepackage{algorithm}
\usepackage{algorithmic}

% Caratula (Recordar logo_uba.jpg y logo_dc.jpg)
\usepackage{caratula}

% Paquetes para tablas
\usepackage[table]{xcolor}

% Se pueden sacar?
\usepackage{url}
\usepackage{float}
\usepackage{afterpage}
\usepackage{tabularx}

% Color de links
\usepackage{hyperref}
\hypersetup{
    colorlinks,
    citecolor=black,
    filecolor=black,
    linkcolor=black,
    urlcolor=black
}

\lstset{
    language=C++,
    basicstyle=\ttfamily,
    breaklines=true,
    breakatwhitespace=true,
    inputencoding=utf8,
    extendedchars=true
    }
\lstset{literate={::}{}{0\discretionary{::}{}{::}}% line-break at ::
   {->}{}{0\discretionary{->}{}{->}}% line-break at ->
}

\begin{document}


\materia{Metodos numericos}
\submateria{Primer Cuatrimestre de 2016}
\titulo{Re entrega Trabajo Pr\'actico 2}
\subtitulo{“CSI:DC”}
\integrante{ Ricardo Colombo}{156/08}{ricardogcolombo@gmail.com}
% keywords
\providecommand{\keywords}[1]{\textbf{\textit{Palabras Claves---}} #1}

\maketitle
\pagebreak
\begin{abstract}
 A lo largo de este trabajo abarcaremos distintas tecnicas y estrategias utilizadas en machine learninng intentando dar con la mas adecuada para obtener la mejor clasificacion de un conjunto de digitos manuscritos basandonos en la informacion mas relevante de cada una de ellas con el fin de poder realizar un reconocimiento de dichos digitos a partir de imagenes que los representan.

\end{abstract}

\keywords{Machine Learning, Reconocimientos de digitos, K vecinos mas cercanos, Analisis de componentes principales. Regresion de minimos cuadrados}


\tableofcontents

\pagebreak

El objetivo de este trabajo es la realización y el análisis de algoritmos eficientes para el reconocimiento óptico de caracteres (OCR), particularmente de dígitos,  a través de la utilización de técnicas simples de Machine learning.
\\
El trabajo consiste en una serie de experimentaciones. El desarrollo de estas encuentra un hilo conductor en las mejoras aplicadas a un algoritmo basadas en problemas particulares que se pueden encontrar en la resolución del problema:

\begin{itemize}

    \item Se parte de una base de datos de imágenes ya etiquetadas y otra con imágenes sin etiquetar. Usando la base de datos etiquetada como información de entrenamiento del algoritmo, se intenta etiquetar de modo correcto los dígitos de la base de datos sin etiquetas.

    \item La primera aproximación a la resolución del problema utiliza el método más intuitivo encontrado: Por cada imagen de la base de datos sin etiquetas, se busca la que más se le parece en la base de datos etiquetada y se marca a la imagen sin etiqueta con la etiqueta de aquella que denominamos como la más parecida. Por supuesto, todavía queda determinar cual es el criterio para decir que dos imágenes se "parecen". Esta definición está dada con profundidad en la sección de desarrollo.

    \item Surge entonces la pregunta acerca de que pasa si, por una particularidad de la imagen, la etiqueta más parecida no es la correcta para el dígito a averiguar. Para mitigar este problema parcialmente se pueden tomar las $k$ imágenes más parecidas (que a partir de ahora llamaremos vecinos) y elegir como etiqueta aquella que se repita más entre los $k$ vecinos. Detrás de esta idea se encuentra el algoritmo $KNN$, que se utiliza para mejorar el comportamiento en estos casos donde el vecino más cercano no pertenece necesariamente a la misma clase que la imagen a etiquetar.

    \item A esta idea se le puede aplicar una mejora sustancial utilizando un método probabilístico conocido como $PCA$. Este consiste en aplicar una transformación a las imágenes, de tal manera de solo tener en cuenta aquellas de mayor variabilidad y desechar aquella información que pueda estar introduciendo ruido.

    \item Por ultimo,  con una idea a $PCA$,  utilizaremos el metodo $PLS-DA$ con la diferencia de utilizar informacion original para realizar la transformacion.


\end{itemize}
Para entender las diferencias y similitudes entre los métodos y sus variantes, se realizan los experimentos con variaciones en los parámetros. En el caso de
$KNN$ se varía la cantidad de vecinos, esto ayuda a entender que valores ayudan a la optimización del algoritmo.
\\
Para el caso de la mejora utilizando el algoritmo de $PCA$ también hay que tener en cuenta el $\alpha$ utilizado. Vamos a ver como modificar este valor
conlleva diferentes tiempos de ejecución y pérdida o ganancia de precisión, y en cuanto al metodo de $PLS-DA$  vamos a variar el valor de gamma.




\pagebreak
Para resolver el enunciado planteado, realizamos la implementación de dos técnicas de calculo de ranking distintas. \\

\subsection{Porcentaje de Victorias}

La primer técnica es \textbf{Porcentaje de Victorias} que a lo largo del análisis denominaremos \textbf{WP} que consiste en tomar el promedio de partidos ganados / partidos jugados. Esta técnica basicamente analiza la performance de un equipo participante en los partidos jugados. \\

En este caso el score de un equipo no es afectado por la cantidad de partidos y resultados obtenidos de los demás participantes, pero esto si afecta su posición final en el ranking. \\

Esta técnica a priori no aporta mucha informacion respectoa la posibilidad de victoria en el siguiente encuentro y tampoco considera el ranking del rival enfrentado. Ya que todos los partidos valen lo mismo. \\

La implementación consiste en calcular: \sum_{i=1}^n{} \frac{G_i}{T} donde \textbf{n} es la cantidad de partidos jugados, \textbf{G_i} corresponde a partidos ganados y \textbf{T} al total de partidos jugados. \\


\subsection{Método Matriz de Colley}

Para la implementación de esta técnica nos basamos en el paper \textbf{The Colley Matrix Explained}. La cual consiste en plantear un sistema de ecuaciones


\subsubsection{Eliminación Gaussiana}
\subsubsection{Cholesky}




\pagebreak
Para analizar los algortimos implementados vamos a utilizar los archivos test1.in y test2.in provistos por la cátedra y realizar una serie de test que nos permitirán en primer caso encontrar parametros buenos con lo que ejecutarlos y posteriormente evaluar el desempeño de la implementación mediante las métricas propuestas por la cátedra.

Dividimos la experimentación en dos secciones, una para cada archivo de prueba. Y evaluaremos los parametros individualmente para cada una de ellas


\subsubsection {Archivo Test1.in}

\subsubsubsection {Algoritmo de K-NN}

Lo primero que vamos a hacer es encontrar un valor de $K$ que nos permita maximizar la cantidad de aciertos, sin tener en consideración las métricas.

Ejecutamos el algoritmos de $K-NN$ variando los valores de k entre {1..20} dejando fija la cantidad de particiones en 10. Luego para cada K nos quedamos con los resultados de la iteración con mas aciertos.

Expresamos los aciertos para cada K en con el siguiente gráfico:



Aca va el grafico.

Como se puede observar la iteración que mas aciertos dió es para $K = 3$. Además se puede observar que a medida que se incrementa el valor de $K$ la cantidad de aciertos va disminuyendo levemente, cumpliendo lo mencionado en la introducción teórica. Cuanto mas corta sea la distancia de los vecinos, mas chances hay de tener un acierto.

Lo siguiente es evaluar fijando el $K$ si el valor de la cantidad de particiones influye en en la cantidad de aciertos. Entonces para eso variamos el número de variamos el número, entre {10..40} incrementando en 10 y tomamos la cantidad de aciertos.

Aca va el grafico.

Como se puede observar la cantidad de particiones no es decisiva a la hora de obtener resultados, ya que no incrementa de forma considerable la cantidad de aciertos. Lo que se observa es que a mayor cantidad de repeticiones, las chances de tener un valor de acierto relativamente mayor se incrementan, esto es solamente debido a que el experimento se repite mas veces y no representa una mejora en la calidad de los resultados.

Caracterizados los valores de $K$ y $Cantidad de Particiones$ que nos dieron mejor cantidad de aciertos. Presentamos las métricas para el mejor caso de cada una de ellas.

Aca iria en forma de tabla y su explicacion


\subsubsubsection {Algoritmo de K-NN con Optimización de PCA}

\subsubsubsection {Algoritmo de K-NN con Optimización de PSL-DA}
Habiendo fijado k = 3 , corremos el test1.in variando el gamma utilizando los valores 1,2,10 y 50. Aqui el grafico
\begin{figure}[H]
\centering
\includegraphics[width=1\textwidth]{chart.jpeg}
\caption{Comparacion de aciertos variando el gamma}
\label{fig:Comparacion de tecnicas}
\end{figure}


Vemos que si aumenta el gamma , mejora la precision pero si gamma aumenta demasiado , en algun momento empeora tu hit rate , suponemos que eso se debe a que si bien tenemos bastante informacion , la cantidad de vecinos no permite aprovecharla . Eso hace suponer que para que tengamos un buen hit rate , debe haber algun tipo de relacion entre el k y el gamma . Eso se va a poner a prueba usando el test2.in

\subsubsection {Archivo Test2.in}

\subsubsubsection {Algoritmo de K-NN}

\subsubsubsection {Algoritmo de K-NN con Optimización de PCA}

\subsubsubsection {Algoritmo de K-NN con Optimización de PSL-DA}

El experimento que nos planteamos es el siguiente. Dejamos fijo el gamma en 13 y hacemos variar el k . Aqui el grafico  
\begin{figure}[H]
\centering
\includegraphics[width=1\textwidth]{chart(2).jpeg}
\caption{Comparacion de aciertos variando el k}
\label{fig:Comparacion de tecnicas}
\end{figure}

De 3 a 7 mejora el hit rate pero es muy poca la mejora en comparacion al aumento de tiempo de computo





\pagebreak
\subsection{Conclusiones}


Luego de la experimentación y análisis de los resultados, concluimos el método de calculo basado en \textbf{CMM} es mas justo en el caso de torneos donde los equipos no juegan la misma cantidad de partidos y donde el empate no es una opción. Ya que asigna un puntaje en base no solo a los resultados obtenidos, sino contra quien fueron obtenidos. Obteniendo un ranking basado en la meritocracia del resultado. \\

Para el caso de torneos donde cada equipo juegue la misma cantidad de partidos el método de \textbf{WP} a nuestro criterio resulta mas justo. Debido a que todos se pusieron a prueba la misma cantidad de veces. \\

Respecto a que implementación de \textbf{CMM} resulta mas eficiente. La conclusion es que depende. Ambas obtienen el mismo resultado, la principal ventaja de \textbf{Cholesky} es que realiza menos computos, mientras que la de \textbf{Eliminación Gaussiana} es que es mas sencilla su implementación. \\

Por último sobre \textbf{La utilización de técnicas avanzadas de análisis de datos son imprescindibles para mejorar cualquier deporte}, consideramos que la frase no es del todo cierta. Afortunadamente la frialdad de los números no es aplicable a la pasión de todos los deportes. Mientras en contados deportes el resultado puede predecirse de antemano, debido a las caracteristicas de los rivales, como en el caso del Polo, esta analogía no puede aplicarse a deportes como el Fútbol donde en innumerables ocasiones el equipo menos favorito termina llevandose el partido.



\pagebreak
 \section{Bibliografía}

\begin{itemize}
 \item Numerical Analysis, Richard L. Burden \& J. Douglas Faires, Chapter 6: Direct Methods for Solving Linear Systems.
\item Machine Learning, Tom M. Mitchell.
\end{itemize}

% \pagebreak
% \section{Codigo}
% \subsection{Sobre los archivos e implementacion}

Para la implementacion de los archivo se utilizo C++, la siguiente es la lista de archivos y consecutivo a la misma hay una descripcion sobre los distintos archivos.

\begin{enumerate}
\item ../src/main.cpp- este contiene la lectura de los archivos de entrada y escritura de la salida, asi como le ejecucion de cada metodo
\item ../src/instancia/instancia.h - clase instancia, una instancia esta compuesta por la matriz CMM , el vector B, una matriz con los partidos ganados del equipo i al equipo j en la posicion (i,j), un arreglo con el total de los partidos. y las definiciones de los metodos para la clase
\item ../src/instancia/instancia.cpp - este archivo contiene todas las implementaciones de los metodos, tanto para generar las matrizes  como los setters y getters de los partes privadas de la clase instancia. 
\item ../src/matriz/matriz.h - Definicion clase matriz, con metodos de get y set y definicion de sus partes privadas y publicas.
\item ../src/matriz/matriz.cpp -Aquise encuentra la implementacion de los metodos de la matriz.
\item ../src/eliminaciongauss/elimgauss.h 
\item ../src/eliminaciongauss/elimgauss.cpp - aquise encuentra la implementaciond de EG
\item ../src/cholesky/cholesky.h 
\item ../src/cholesky/cholesky.cp  - Aqui se encuentra la implementacion de la factorizacion de Cholesky.
\item ../src/wp/wp.h
\item ../src/wp/wp.cpp - Aqui se encuentra la implementacion del metodo WP
\end{enumerate}

La clase instancia la definimos para que sea mas facil el manejo de una instancia en general de juego, 
en base a su matriz CMM , matriz de partidos ganados y vector b, con fin de facilitarnos el uso de la entrada.

En cuanto a la implementacion del clase matri se utilizo un puntero a double donde en cada posicion hay otro puntero a double, 
ademas definimos setters y getters para las posiciones para que sea mas facil si uso y modularizar cada parte del programa, como los algoritmos relevantes a 
Eliminacion gaussiana, Cholesky y WP para una mas facil lectura.

\subsection{Codigo implementado}
\lstinputlisting[language=C++]{../src/main.cpp}
\lstinputlisting[language=C++]{../src/matriz/matriz.h}
\lstinputlisting[language=C++]{../src/matriz/matriz.cpp}
\lstinputlisting[language=C++]{../src/instancia/instancia.h}
\lstinputlisting[language=C++]{../src/instancia/instancia.cpp}
\lstinputlisting[language=C++]{../src/eliminaciongauss/elimgauss.h}
\lstinputlisting[language=C++]{../src/eliminaciongauss/elimgauss.cpp}
\lstinputlisting[language=C++]{../src/cholesky/cholesky.h}
\lstinputlisting[language=C++]{../src/cholesky/cholesky.cpp}
\lstinputlisting[language=C++]{../src/wp/wp.h}
\lstinputlisting[language=C++]{../src/wp/wp.cpp}



\end{document}
